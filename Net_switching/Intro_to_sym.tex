\documentclass{article}[12 pt]

\title{Introducción a la simulación de redes de comunicaciones}
\author{Carlos Ortega Marchamalo \& Pablo Collado Soto}
\date{}

\begin{document}

	\maketitle

	\section{Análisis del enlce Londres - Dublín}
		Tras configurar \texttt{Comnet III} para llevar a cabo la simulación deseada con el escenario proporcionado fuimos capaces de obtener una gran cantidad de datos "crudos" que fueron posteriormente tratados con \texttt{Excel}. Para poder discutir la validez de las mediciones debemos tener un modelo con el que compararlas con lo que necesitamos estudiar los parámetros que definen nuestro escenario. En este caso el sistema es un enlace de datos con una tasa binaria fija de $R = XXX\ \frac{kbit}{s}$. Podríamos pensar directamente que la mejor forma de caracterizar el sistema es a través de un sistema de colas $M/D/1$, es decir, aquel en el que los tiempos entre llegadas se distibuyen exponencialmente con un tiempo de servicio constante (o determinista, de ahí la \texttt{D}). Además, solo podremos atender a un cliente de manera simultánea sin tener la posibilidad de perder a ninguno pues en ausencia de un parámetro explícito la notación de \textit{Kendall} nos permite asumir que el tamaño total de todo el sistema es $\infty$ con lo que la cola de espera para ser atendido puede crecer sin límite.
		\\
		Hemos sido ambiguos al introducir el sistema de manera intencionada porque queremos dejar clara la independencia de un modelo $M/D/1$ de nuestro escenario particular. Para nosotros los clientes serán paquetes de información, el servidor será el propio enlace y el tiempo de servicio será el que el paquete tarde en llegar a su destino. Asimismo, la cola se generará dentro del router conectado a dicho enlace. Es cierto que la memoria de estos equipos es finita con lo que las colas no serán infinitamente grandes pero, dada una situación favorable podremos asumir esta premisa. Si bien es verdad que no todos los sistemas $M/D/1$ son enlaces de datos los enlaces de datos sí pueden caracterizarse a través de una $M/D/1$. La pregunta que nos atañe entonces es, ¿pueden todos los enlaces caracterizarse como modelos $M/D/1$? Veremos que, al contrario de lo que podríamos pensar, esto \textbf{NO} siempre será cierto... Si asumimos que el tiempo de servicio del enlace es determinista podemos estar incurriendo en un fallo sin darnos cuenta. Para caracterizar el modelo debemos estudiar el tiempo de servicio y, si bien la tasa binaria del enlace $R$ es constante ésta no define por completo el tiempo de servicio. Recordemos que el tiempo de transmisión de un paquete de longitud $L$ a través de un enlace con una tasa binaria $R$ viene dado por $t_{trans} = \frac{L}{R}$. Recordando el significado de este tiempo de transmisión vemos que marca la capacidad del enlace en cuanto a la cantidad de datos que puede manejar por unidad de tiempo. Al no tener en cuenta el retardo de la propia información por el enlace vemos pues que la "imagen" no está completa sin hablar del retardo de propagación. \textit{Albert Einstein} demostró que nada es más rápido que la luz que viaja a nada menos que $c \approx 3 \cdot 10^8 \frac{m}{s}$. Ésto implica que los "bits" que volcamos al enlace tardarán un tiempo $t_{prop}$ en llegar al otro extremo. Por tanto, si somos estrictos el tiempo de servicio de cada paquete vendrá dado por $t_{s} = t_{prop} + t_{trans}$. Teniendo el corazón de electrónicos que tenemos podemos pensar que esta expresión es siempre determinista, es decir, los paquetes siempre tendrán la misma longitud con lo que $t_{trans} = cte$. Si dejamos que entre nuestra cabeza de telemático veremos que esto \textbf{NO} es así en nuestro caso... Para comprender qué está ocurriendo tenemos que bucear en las opciones del simulador para ver qué tipo de paquetes se están generando...
		\\

\end{document}
