\documentclass{article}

\usepackage{geometry}
\geometry
{
 a4paper,
 left = 30mm,
 right = 30 mm,
 top = 20 mm
}
\title{Informe individual Seguridad}
\author{Carlos Ortega Marchamalo}
\date{}

\begin{document}

	\maketitle

		\section{Ataque de diccionario}
			En cuanto al archivo \textit{raw-md5.hashes4.txt} observamos que se han recuperado 156806 contraseñas en un tiempo de 9.7 s, es decir, en torno a un 4.5 \% del total también.

		\section{Ataque de combinación}
			Haciendo uso del ataque de diccionario se nos presenta una limitación muy importante pues simplemente probamos con las contraseñas contenidas en ese diccionario tal y como aparecen en él por lo que no se tienen en cuenta posibles claves que consisten en combinaciones o concatenaciones de esas contraseñas incluidas en el diccionario. Es por ello que hemos decidido a dar un paso más y tener esta posibilidad presente.

			Así, llevamos a cabo el ataque de Combinación el cual se apoya en dos diccionarios para construir las nuevas contraseñas, encadenando cada una de las palabras de uno con cada una del otro. Como consecuencia, se llegan a probar hasta $3107 * 3107 = 9653443$ contraseñas, o lo que es lo mismo, el esfuerzo máximo es de 9653443 hashes pues el diccionario, que es empleado dos veces, consta de 3107 palabras.

			Para tener una mejor perspectiva de los beneficios de utilizar este método hemos recurrido antes al ataque de diccionario propiamente dicho, con el que hemos obtenido un resultado de \textbf{36} contraseñas rotas en un tiempo de 10.198 s. Para esto hemos seguido el mismo procedimiento que en el ataque de diccionario mencionado previamente con la variación de que en esta ocasión el archivo de hashes a auditar es \textit{raw-md5.hashes4.txt}.

			Posteriormente ponemos en marcha el ataque de combinación, estableciendo para ello la opción \textit{-a} con el valor \textit{1}. Por consiguiente, el comando resultante a lanzar es \textit{time hashcat -m 0 -a 1 --potfile-disable raw-md5.hashes4.txt john.txt john.txt}.

			Observamos como, en tan solo 11.207 s, se han recuperado hasta \textbf{39418} contraseñas. De este modo apreciamos los enormes beneficios y la inmensa diferencia entre usar un diccionario directamente y combinar entre ellas las palabras que en él se encuentran.

			Estudiando la información que se muestra tras concluir el ataque vemos como, efectivamente, se han probado hasta 9653443 contraseñas.

			Contemplando las posibles claves a probar advertimos como se intentará, por ejemplo, con \textit{michaeljordan}. En este caso podemos comprobar como se ha llegado a romper la contraseña \textit{adidaschelsea}, encontrándose cada una de estas dos palabras en el diccionario considerado.
\end{document}