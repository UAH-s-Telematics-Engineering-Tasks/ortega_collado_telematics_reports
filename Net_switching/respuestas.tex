4.
Las llamadas realizadas desde Dublín únicamente poseen como destino Madrid. En referencia a ellas, podemos observar mediante el análisis y el estudio del fichero "reports" generado de forma automática por el programa en el cual nos apoyamos que presentan una probabilidad de bloqueo del 0.132.

5.
La variación en el parámetro 1/lambda influye directamente en el factor de ocupación del canal, siendo inversamente proporcional esta relación. En nuestro caso, con el transcurso de los experimentos se produce una disminución en el porcentaje de utilización del canal. Este cambio se produce de forma similar en ambos sentidos.

En cambio, en lo que se refiere al retardo de los mensajes se produce una circunstancia particular, pues en el sentido L-D permanece constante. En el otro sentido sí que se producen variaciones, siendo estas del orden de algún milisegungo o, de forma equivalente, algunos cientos de microsegundos.

6.
