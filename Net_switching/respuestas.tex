4.
Pese a existir diveros posibles destinos de las llamadas realizadas desde Dublín pues esta sede posee conexión tanto con el nodo localizado en Madird así como con el localizado en Londres, llevando a cabo un estudio minucioso de la información obtenida navegando por los distintos menús disponibles en el punto donde se generan las llamadas podemos circiorarnos con total seguridad de que estas únicamente poseen como destino Madrid pues así resulta explicitado en uno de los parámetros de configuración observados. El modelo empleado para poder llevar a cabo un estudio más sencillo del comportamiento de todo el esquema se corresponde con el de un sistema M/M/N/N, tomando como valor para el número de líneas o canales dedicados el más limitante, en nuestro caso 14, pues este es el valor que antes imposibilita el transcurso de nuevas llamadas ya que, aunque la central que en primer lugar recibe las llamadas dispone de capacidad para cursar un número mayor, si no existen recusos desde ella hacia el exterior las solicitudes no pueden transcurrir, lo que explica la selección del valor que establece el tope en la capacidad del sistema en conjunto. En referencia a estas llamadas generadas en Dubín y con destino Madrid, el análisis y el estudio del fichero "reports" generado de forma automática por el programa COMNET III en el cual nos apoyamos para poder desarrolar todo este estudio nos proporciona un apartado específico referente a las probabilidades de bloqueo que sufren las llamadas que tiene lugar en el sistema y que son objeto de nuestra investigación. De este modo, poniendo el foco de atención en las que en concreto tienen su origen en Dublin nos percatamos de que presentan una probabilidad de bloqueo del 0.132, un valor en cierto modo coherente y concordante con los cálculos matemáticos llevados a cabo en función del sistema planteado en primera instancia, es decir, la probabilidad de bloqueo para un sistema M/M/N/N siendo N = 14 líneas y presentando las llamadas un tiempo de generación (1/lambda) de 30 segundos así como un tiempo de duracón de 3 minutos, ambos parámetros siguiendo un modelo de distribución exponencial y resultando el tráfico generado en cada sede de lambda*1/u igual a 6 Erlangs, con lo que en total se tiene un trafico de 12 Erlangs pues debemos considerar los dos sentidos pues ambos transcurren por las mismas líneas, las 14 mencionadas.

5.
La variación en el parámetro 1/lambda influye directamente en el factor de ocupación del canal, siendo inversamente proporcional esta relación. En nuestro caso, con el transcurso de los experimentos se produce una disminución en el porcentaje de utilización del canal. Este cambio se produce de forma similar en ambos sentidos.

En cambio, en lo que se refiere al retardo de los mensajes se produce una circunstancia particular, pues en el sentido L-D permanece constante. En el otro sentido sí que se producen variaciones, siendo estas del orden de algún milisegungo o, de forma equivalente, algunos cientos de microsegundos.

6.
