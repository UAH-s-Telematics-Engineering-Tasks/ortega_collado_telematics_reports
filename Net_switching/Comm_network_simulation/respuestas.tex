4.
Para poder llevar acabo una análisis más preciso y que se ajuste lo máximo posible a la realidad exiten dos alternativas. Por un lado y tal y como hemos venido realizando en los casos anteriores, podemos aumentar el número de repeticiones realizadas sobre la simulación. En el caso que ahora nos atañe vamos a hacer uso de la otra vía de la que disponemos para lograr el fin deseado. Así, aumentando el tiempo de duración de la repetición logramos recopilar un número mayor de datos lo que nos permite establecer una relación con lo que sucede realmente.

En el escenario del que disponemos, pese a existir diveros posibles destinos de las llamadas realizadas desde Dublín pues esta sede posee conexión tanto con el nodo localizado en Madird así como con el localizado en Londres, llevando a cabo un estudio minucioso de la información obtenida navegando por los distintos menús disponibles en el punto donde se generan las llamadas podemos circiorarnos con total seguridad de que estas únicamente poseen como destino Madrid pues así resulta explicitado en uno de los parámetros de configuración observados.

El modelo empleado para poder llevar a cabo un estudio más sencillo del comportamiento de todo el esquema se corresponde con el de un sistema $M/M/N/N$, tomando como valor para el número de líneas o canales dedicados el más limitante, en nuestro caso 14, pues este es el valor que antes imposibilita el transcurso de nuevas llamadas ya que, aunque la central que en primer lugar recibe las llamadas dispone de capacidad para cursar un número mayor, si no existen recusos desde ella hacia el exterior las solicitudes no pueden transcurrir, lo que explica la selección del valor que establece el tope en la capacidad del sistema en conjunto.

En referencia a estas llamadas generadas en Dubín y con destino Madrid, el análisis y el estudio del fichero "reports" generado de forma automática por el programa \texttt{COMNET III} en el cual nos apoyamos para poder desarrolar todo este estudio nos proporciona un apartado específico referente a las probabilidades de bloqueo que sufren las llamadas que tiene lugar en el sistema y que son objeto de nuestra investigación.

De este modo, poniendo el foco de atención en las que en concreto tienen su origen en Dublin nos percatamos de que presentan una probabilidad de bloqueo del 0.132, un valor en cierto modo coherente y concordante con los cálculos matemáticos llevados a cabo en función del sistema planteado en primera instancia, es decir, la probabilidad de bloqueo para un sistema $M/M/N/N$ siendo N = 14 líneas y presentando las llamadas un tiempo de generación (1/lambda) $1/\lambda$ de 30 segundos así como un tiempo de duracón de 3 minutos, ambos parámetros siguiendo un modelo de distribución exponencial y resultando el tráfico generado en cada sede igual a (lambda*(1/nu)) $\lambda*(1\mu)$, es decir, 6 Erlangs, con lo que en total se tiene un trafico de 12 Erlangs pues debemos considerar los dos sentidos ya que ambos transcurren por las mismas líneas, las 14 mencionadas.

Haciendo uso de la calculadora de tráfico de la que disponemos e introduciendo los valores citados anteriormente correspondientes al tráfico total así como el número de líneas disponibles obtenemos un valor bastante próximo al observado mediante los ficheros "reports" generados con el simulador que empleamos. En nuestro caso, obtenemos una probabilidad de bloqueo teórica de 0.117, un valor realitivamente cercano al anterior y que difiere e forma pequeña en las imperfecciones presentadas por el modelo empleado para tratar de imitar el funcionamiento de todo el conjunto en global.

En conclusión, nos percatamos de que se está produciendo un bloqueo de un porcentaje importante de las llamadas que se originan en la sede de Dublín con destino Madrid. Se trata de algo más del $10\%$ más específicamente, un valor un tanto elevado para lo que sería un valor apropiado y deseado en la realidad, en la que se busca reducir al mínimo esta probabilidad, siendo prácticamente de obligación situarla como máximo en una cifra porcentual, es decir, un valor inferior a la decena y siempre lo más cercano a 0.

5.
El programa en el cual nos apoyamos para llevar a cabo un análisis más detenido del transcurso del tráfico tanto de paquetes como de llamadas entre varios nodos, \texttt{COMNET III}, nos permite ir un poco más allá en este estudio y realizar incluso simulaciones sucesivas que difieran en un parámetro en concreto, el cual sufre unas variaciones explicitadas de antemano, lo que posibilita conocer la influencia y transcendencia del mismo en el funcionamiento global de todo el sistema que nos ocupa.

De este modo, en nuestro caso hemos decidido realizar una serie de simulaciones centrándonos en la trascendencia que presenta el tiempo entre llegadas de mensajes, es decir, (1/lambda) $1/\lambda$, en cada uno de los nodos que son capaces de generar esta información, es decir, tanto en las sedes de Dublín como de Madrid.

En todas las variantes del experimento a realizar el parámetro de interés sigue una distribución exponencial, la cual se encuentra caracterizada en cada simulación por uno de los valores incluidos en la lista especificada al llevar a cabo la configuración previa del escenario correspondiente a través de los distintos menús disponibles. Estos datos en concreto son 0.5, 1.0, 2.0, 4.0, 6.0, 8.0 y 10.0, respectivamente, para cada uno de los 7 experimentos de los que consta la cuestión que a estudiar.

Llevando a cabo esta cnfiguración basada en un dato que varía de una simulación a otra podemos analizar cómo responde nuestro esquema ante los diferentes escenarios que tienen lugar, percibiendo qué sucede en los enlaces cuando los paquetes se generan con una frecuencia mayor o cómo a medida que este ritmo de generación disminuye la congestión así también lo hace. De forma análoga ocurre con el retardo que sufren los paquetes desde que son generados hasa que llegan al destino, circunstancia que también podremos apreciar gracias a las particularidades del caso que abordamos.

Tras poner en marcha la prueba y esperar a la conclusión de la misma, nos es posible obtener diversos documentos de estadísticas así como ficheros "reports", resultando más precisos los primeros y ajustándose en mayor medida a los valores obtenidos tras llevar a cabo el correspondiente análisis matemático, el cual, mediante el parámetro (ro = lambda/nu) $\rho = \lambda/\mu$ nos informa del factor de utilización u ocupación teórico que debería presentar el canal por el que transcurre todo el tráfico.

Al establecer el punto de mira en los valores obtenidos y plasmados en el documento de estadística apreciamos que la variación en el parámetro (1/lambda) $1/\lambda$, es decir, el tiempo entre llegadas de mensajes, influye de manera inversamente proporcional en el factor de ocupación del canal. Disminuyendo la frecuencia con la quese generan los paquetes, o lo que es lo mismo, aumentando el tiempo entre la llegada de cada paquete, logramos que cada vez vaya disminuyendo más el número de paquetes lo que descongestiona el canal por el que circulan los mismos. De forma inversa sucedería de manera similar, por lo que una reducción del tiempo entre cada mensaje aumentaría la ocupación del medio transmisor.

En nuestro caso, con el transcurso de los experimentos se produce una disminución en el porcentaje de utilización del canal. Este cambio se produce de forma similar en ambos sentidos.

Pero la importancia de la fluctuación del parámetro explicitado no se queda aquí. Como era de esperar, su influencia va más allá y presenta su transcendencia en cuanto al tiempo que tardan los paquetes en llegar desde su orígen hasta su destino. Esto tiene su lógica pues cuanto mayor sea la congestión que presente el canal, más tiempo debe esperar el paquete para viajar a través de él.

Este hecho puede apreciarse en cierto modo y un tanto de forma pequeña al observar los tiempos transcurridos en el sentido (desde Dublín hasta Londres) \texttt{Dublín --> Londres} mediante las estadísticas disponibles. Sin embargo, la disminución apreciada resulta en cierta medida más bien mínima, siendo esta del orden de algún milisegungo o, de forma equivalente, algunos cientos de microsegundos.

En cambio, no sucede de forma similar en el sentido contrario, es decir, (desde Londres hasta Dublín) \texttt{Londres --> Dublín}. Esto es debido, tal y como hemos comentado en cuestiones anteriores, a que los paquetes que viajan en esta dirección presentan un tamaño constante y, además, pasan de un enlace más lento a uno más rápido por lo que en el segundo no se forman colas y los tiempos permanecen constantes y invariantes.

A través de este caso hemos podido constatar la transcendencia que presenta así como la influencia de la variación de un parámetro del sistema global como es el caso del tiempo entre llegadas de mensajes para el grado de ocupación o utilización del canal por el que transcurren los paquetes y su consiguiente relevancia en los retardos sufridos por estos últimos al viajar por el medio transmisor.

6.
Llegado a este punto, procedemos a incidir un ápice más en la situación que se nos había presentado en un primer momento. En ella se tiene un sistema caracterizado por un parámetro (alpha = 0.1) $\alpha = 1$ el cual define los intervalos de confianza sobre cada una de las mediciones en torno al modelo llevadas a cabo.

Para poder extraer una conclusión más precisa y que se asimile lo máximo posible a la realidad con anterioridad decidimos llevar a cabo una simulación incrementando el número de repeticiones que se iban a producir, llevando esta cifra hasta el valor de 40 lo que nos permitía obtener un intervalo de confianza más reducido y acotado.

Como comentamos en primera instancia en el momento de realizar un análisis un poco más profundo del esquema del que disponemos, el enlace existente entre Londres y Dublín puede asimilarse a un modelo de colas $M/D/1$ por cuanto los mensajes que son generados en el origen presente una longitud fija y constante que no sufre variaciones en ningún momento por lo que el resultado que produce es siempre el mismo, resultando también fijo el tiempo que tarda en atravesar un paquete el enlace establecido entre los routers de las dos localizaciones citadas previamente. Por su parte, en el sentido contrario el escenario resulta un tanto diferente ya que se corresponde con un sistema $M/M/1$ pues la longitud de los paquetes sigue una distribución exponecial, lo que se traduce en cambios en la misma y, de forma consecuente, en los tiempos que se pueden medir respecto a la disposición existe.

En este último apatado de estudio la variación presentada respecto a lo presentado en un primer momento reside en la modificación de la capacidad que poseen los enlaces que unen los routers presentes entre si. Este valor de la tasa binaria pasa a incrementarse hasta los 256kbps, hecho que tendrá su transcendencia y que procedemos a comprobar y a analizar.

Si tomamos el documento de estadísticas que se genera después de la ejecución del programa y tras llevar a cabo el consiguiente filtrado de la múltiple información que presenta, nos centramos en el retardo que sufren los mensajes en el enlace situado entre los routers de Dublín y Londres así como en el retardo de los paquetes entregados por la fuente de mesajes de Dublín.

Analizando estos datos, su concordancia en cuanto a los calculos matemáticos también resulta correcta, de forma similar a lo que sucedía en las primeras cuestiones que abordamos en el documento que nos ocupa.

Llevando a cabo una comparación entre los resultados obtenidos en este último caso con los que se obtuvieron en las simulaciones iniciales podemos apreciar ciertas variaciones, tal y como era de esperar. Esta suposición previa es debida a que ambas situaciones difieren en la tasa binaria del enlace, con lo que nos encontramos con algo que preveíamos.

Profundizando más en los valores recolectamos percibimos que al aumentar el parámetro que varía conseguimos reducir los tiempos resultantes. Las mediciones referentes al enlace entre los router de Dublín y Londres en ambos sentidos se ven disminuidos a la mitad, algo en cierto modo de esperar por cuanto la capacidad del medio se ha visto incrementada también duplicándose. Respecto al tiempo retardo de los paquetes entregados por la fuente de mesajes de Dublín esta medición sufre una disminución de forma similar, aunque esta no resulta tan abultada ya que no solo depende del enlace que ha aumentado su régimen binario si no que implica a otros enlaces más que son trascendentes al ser más lentos, lo que marca el resultado final.

Respecto a los intervalos de confianza, estos se ven más concretados y reducidos, estableciendo unos valores más acotados y precisos.

De este modo, podemos observar la trascendencia de modificar un valor y poniendo el foco en un punto en concreto, donde podemos apreciar cambios significativos, pero que, al generalizar para todo el esquema, no tienen una influencia tan grande por cuanto son más los elementos involucrados en el cómputo global.

A modo de resumen para este último caso, incluímos una tabla para poder visualizar más explícitamente esta variaciones comparando ambos casos. Los cálculos matemáticos reaizados para corroborar lo visualizado a través de la simulación del escenario definitivo son exactamente iguales al precedente anterior con la excepción de la variación de la tasa binaria.

\begin{tabular}{| c | c | c | c | c | c | c |}
			\hline
			$40$ iteraciones; $[ms]$ & Media & L. Inferior & L. superior & Int. Confianza & Media Teórica & Media 3er ejercicio\\
			\hline
			Retardo $B$ \texttt{D-L} & 28,7 & 26,8 & 30,61 & 1,91 & 31,45 & 57,41\\
			\hline
			Retardo $B$ \texttt{L-D} & 6,25 & 6,25 & 6,25 & 0 & 0 & 12,5\\
			\hline
			Retardo pkts \texttt{D-L} & 277,082 & 258,473 & 295,69 & 18,608 & 313,82 & 356,914\\
			\hline
\end{tabular}

La realización de la práctica que nos ocupa nos he permitido profundizar mucho más en diversos aspectos de las redes de comunicación tanto de datos como de voz de los cuales ya éramos conocedores pues los habíamos estudiado en diversas asignaturas a lo largo de nuestro grado universitario, pero sobre los que, realmente, no llegábamos a asimilar su verdadera influencia e importancia en todo el entramado que constituyen las mencionadas redes.

		Así, hemos podido experimentar con la modificación de parámetros que caracterizan tanto los paquetes que viajan a nuestro alrededor como las llamadas que se producen entre la población, además de los enlaces por los cuales transcurre toda esta información.

		Existen diversas circunstancias muy importante para poder relacionar con mayor precisión las simulaciones llevadas a cabo con la estricta realidad. Entre ellas sobresalen dos, las cuales han sido puestas en práctica con instancia a lo largos de los diferentes escenarios planteados relacionados con el modelo de red proporcionado. La primera de ella se basa en aumentar el número de repeticiones que deben realizarse sobre el escenario conformado. De forma similar, incremetar el tiempo de duración de estas repeticiones también es un acontecimiento que nos ayuda a un lograr un mejor análisis. La razón de todo ello es que se consigue recolectar un mayor número de muestras y datos lo cual posibilita acotar muchos más las mediciones y reducir los intervalos de confianza de las mismas, aspecto de suma importancia de igual modo.

		Este experimento implementado nos ha posibilitado, además, comprender la relación existente entre la realidad y los modelos que se emplean para conectar esta con las puras matemáticas que facilitan todo el estudio numérico existente. Ello nos ha conducido a apreciar los mínimos pero existentes errores que surgen al establecer esta relación. Del mismo modo, la asociación con un esquema de colas nos permite calcular de forma teórica algunas mediciones para compararlas posteriormente con la propia simulación, apreciando además cómo influyen los tipos de distribución estadística en la caracterización de los parámetros implicados, viendo como en algunos casos, por extraño que parezca, los tiempos permanecen constantes, mientras que en otros casos el aumento o disminución de ciertos valores influye muy significativamente en los resultados finales. Hemos podido cerciorarnos de las limitaciones existentes en función de las capacidades de los medios a través de los cuales, al fin y al cabo, viaja toda la información intercambiada y cómo estas influyen en las mediciones e incluso hasta en el consiguiente bloqueo de llamadas.

		Por tanto, la práctica abordada resulta de gran utilidad y curiosidad para comprender un poco más todo lo que sucede en las redes de comunicación existentes a nustro alrededor en cualquier lugar y de las cuales hacemos uso prácticamente en cualquier momento, incluso sin llegar a ser conscientes de ellos y de la importancia y trascendencia de todos y cada uno de sus componentes.