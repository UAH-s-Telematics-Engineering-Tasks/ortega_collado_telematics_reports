4.
Pese a existir diveros posibles destinos de las llamadas realizadas desde Dublín pues esta sede posee conexión tanto con el nodo localizado en Madird así como con el localizado en Londres, llevando a cabo un estudio minucioso de la información obtenida navegando por los distintos menús disponibles en el punto donde se generan las llamadas podemos circiorarnos con total seguridad de que estas únicamente poseen como destino Madrid pues así resulta explicitado en uno de los parámetros de configuración observados. El modelo empleado para poder llevar a cabo un estudio más sencillo del comportamiento de todo el esquema se corresponde con el de un sistema M/M/N/N, tomando como valor para el número de líneas o canales dedicados el más limitante, en nuestro caso 14, pues este es el valor que antes imposibilita el transcurso de nuevas llamadas ya que, aunque la central que en primer lugar recibe las llamadas dispone de capacidad para cursar un número mayor, si no existen recusos desde ella hacia el exterior las solicitudes no pueden transcurrir, lo que explica la selección del valor que establece el tope en la capacidad del sistema en conjunto. En referencia a estas llamadas generadas en Dubín y con destino Madrid, el análisis y el estudio del fichero "reports" generado de forma automática por el programa COMNET III en el cual nos apoyamos para poder desarrolar todo este estudio nos proporciona un apartado específico referente a las probabilidades de bloqueo que sufren las llamadas que tiene lugar en el sistema y que son objeto de nuestra investigación. De este modo, poniendo el foco de atención en las que en concreto tienen su origen en Dublin nos percatamos de que presentan una probabilidad de bloqueo del 0.132, un valor en cierto modo coherente y concordante con los cálculos matemáticos llevados a cabo en función del sistema planteado en primera instancia, es decir, la probabilidad de bloqueo para un sistema M/M/N/N siendo N = 14 líneas y presentando las llamadas un tiempo de generación (1/lambda) de 30 segundos así como un tiempo de duracón de 3 minutos, ambos parámetros siguiendo un modelo de distribución exponencial y resultando el tráfico generado en cada sede igual a (lambda*(1/nu)), es decir, 6 Erlangs, con lo que en total se tiene un trafico de 12 Erlangs pues debemos considerar los dos sentidos ya que ambos transcurren por las mismas líneas, las 14 mencionadas.

5.
El programa en el cual nos apoyamos para llevar a cabo un análisis más detenido del transcurso del tráfico tanto de paquetes como de llamadas entre varios nodos, COMNET III, nos permite ir un poco más allá en este estudio y realizar incluso simulaciones sucesivas que difieran en un parámetro en concreto, el cual sufre unas variaciones explicitadas de antemano, lo que posibilita conocer la influencia y transcendencia del mismo en el funcionamiento global de todo el sistema que nos ocupa. De este modo, en nuestro caso hemos decidido realizar una serie de simulaciones centrándonos en la trascendencia que presenta la tasa de generación de mensajes (1/lambda) en cada uno de los nodos que son capaces de generar esta información, es decir, tanto en las sedes de Dublín como de Madrid. En todas las variantes del experimento a realizar el parámetro de interés sigue una distribución exponencial, la cual se encuentra caracterizada en cada simulación por uno de los valores incluidos en la lista especificada al llevar a cabo la configuración previa del escenario correspondiente a través de los distintos menús disponibles. Estos datos en concreto son 0.5, 1.0, 2.0, 4.0, 6.0, 8.0 y 10.0, respectivamente, para cada uno de los 7 experimentos de los que consta la cuestión que nos ocupa. Tras poner en marcha la prueba y esperar a la conclusión de la misma, nos es posible obtener diversos documentos de estadísticas así como ficheros reports, resultando más precisos los primeros y ajustándose en mayor medida a los valores obtenidos tras llevar a cabo el correspondiente análisis matemático, el cual, mediante el parámetro (ro = lambda/nu) nos informa del "factor de utilización" (comprobar) teórico que debería presentar el canal por el que transcurre todo el tráfico. La variación en el parámetro 1/lambda influye directamente en el factor de ocupación del canal, siendo inversamente proporcional esta relación. En nuestro caso, con el transcurso de los experimentos se produce una disminución en el porcentaje de utilización del canal. Este cambio se produce de forma similar en ambos sentidos.

En cambio, en lo que se refiere al retardo de los mensajes se produce una circunstancia particular, pues en el sentido L-D permanece constante. En el otro sentido sí que se producen variaciones, siendo estas del orden de algún milisegungo o, de forma equivalente, algunos cientos de microsegundos.

6.
